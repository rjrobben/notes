\documentclass{article}

\title{Technical Work-Log}
\author{Raymond Yip}
\date{25 September, 2022}

\begin{document}
\maketitle

This document will be a log of my work. My work includes everything
related to my tech career, e.g.~my paid job, my personal projects,
personal setup, freelance, .etc. My learning experience can appear here
as well but I will set up another folders for storing my learning notes.

\section*{2022-9-25}

Today I have tried debugging vimtex by exploring its source code. The
error I faced is ``viewer cannot find Zathura window ID.'' I have no
idea how I get this error at first. Because I have already installed
Zathura on wsl, (plus a bunch of other sofwares, e.g.~vcsxrv, x11-apps)
and the zathura is working on my wsl. I can open a pdf window in wsl
terminal to view documents in windows. So it must not be related to
Zathura installation problem.

I have tried changing the setting: delete the default
g:vimtex\_view\_method=``zathura'' and change to
g:vimtex\_view\_general\_viewer=``zathura'' in init.lua instead. It
works. I can now use VimtexView to see the pdf file compiled. But
forward search and inverse search don't work.

That' why I decided to inspect the source code of vimtex and see the
line that calls the zathura command. After inspecting the function and
learning about vimscript language, I locate a self.\_start method and
insepct the function called inside this method. After 1-2 hours, I found
the line calling the shell command of zathura and pdf file, which is:

\begin{verbatim}
function! s:cmdline(outfile, synctex, start) abort " {{{1
  let l:cmd  = 'zathura'

  if a:start
    let l:cmd .= ' ' . g:vimtex_view_zathura_options
    if a:synctex
      let l:cmd .= printf(
            \ ' -x "%s -c \"VimtexInverseSearch %%{line} ''%%{input}''\""',
            \ s:inverse_search_cmd)
    endif
  endif

  if a:synctex && (!a:start || g:vimtex_view_forward_search_on_start)
    let l:cmd .= printf(
          \ ' --synctex-forward %d:%d:%s',
          \ line('.'), col('.'),
          \ vimtex#util#shellescape(expand('%:p')))
  endif

  return l:cmd . ' '
        \ . vimtex#util#shellescape(vimtex#paths#relative(a:outfile, getcwd()))
        \ . '&'
endfunction 
\end{verbatim}

So, the problem lies in the flag argument passed to the zathura command
when opening pdf. I have tried to use the same flag:

\begin{verbatim}
zathura --synctex-forward 28:1:hello.tex hello.pdf
\end{verbatim}

It gives an error of course. Bingo! Now the error probably springs from
this command. So i went on to google this error and the solution is
install dbus-x11. Go ahead and all errors are gone.

\section*{2022-09-26}
I always encounter a problem in my workflow:\\

\noindent\fbox{%
    \parbox{\textwidth}{%
		How do I go to some code or webpage or documents or jupyter notebook or pdf or \textbf{any files} that contain information that I know I need from my rough memory. 
    }%
} \newline\\
I have wasted a lot of time on idling for not knowing what to go next, that really reduce my efficiency. 
I think it is necessary to design a workflow that facilitate exactly that. There are a few things that I have to know for such workflow:

\begin{enumerate}
	\item I have to know exactly what I am going to find/do.
	\item Define broad ideas for actions and stuff in my head ram.
	\item eg, for my current job, the basics are: {Plotting}, {Calculate}, {Database}. 
\end{enumerate}

\end{document}
